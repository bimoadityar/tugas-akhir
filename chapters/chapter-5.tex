\chapter{PENUTUP}

Bab ini memaparkan hasil-hasil simpulan yang dapat ditarik berdasarkan uji coba beserta saran yang dapat diambil untuk pengembangan-pengembangan berikutnya.

\section{Simpulan}

Berdasarkan hasil eksperimen yang diperoleh, didapatkan bahwa algoritma estimasi gerakan robot tim dan bola yang paling akurat adalah algoritma yang:
\begin{enumerate}
    \item Menggunakan algoritma AMCL berbasis \textit{pose} dan \textit{twist} untuk mengestimasi gerakan robot sendiri
    \item Menggunakan algoritma penapis partikel dengan penanganan data terlambat untuk mengestimasi gerakan bola
    \item Menggunakan algoritma penapis partikel dengan penanganan data terlambat untuk mengestimasi dan mengoreksi gerakan robot teman
    \item Mengoreksi hasil estimasi berdasarkan selisih waktu sekarang dengan \textit{timestamp} data terakhir
\end{enumerate}

Konfigurasi-konfigurasi tersebut mempunyai kinerja yang lebih baik dibandingkan dengan alternatif seperti menggunakan AMCL berbasis \textit{pose} saja atau estimasi gerakan bola menggunakan penapis Kalman.

\section{Saran}

Saran yang dapat diterapkan untuk pengembangan-pengembangan berikutnya diantaranya adalah:
\begin{enumerate}
    \item Memvalidasi hasil uji coba dengan percobaan langsung di lingkungan nyata dengan robot di lapangan atau menggunakan data yang diambil dari uji coba robot di lapangan
    \item Melakukan koreksi posisi robot teman menggunakan persepsi \textit{vision} terhadap teman tersebut
    \item Menyebarkan seluruh distribusi estimasi dibandingkan hanya estimasinya saja melalui komunikasi
    \item Menerapkan berbagai variasi modifikasi penapis Kalman yang ada
\end{enumerate}

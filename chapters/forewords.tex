\chapter*{Kata Pengantar}
\addcontentsline{toc}{chapter}{KATA PENGANTAR}

Puji syukur penulis panjatkan kepada Allah SWT karena hanya dengan rahmat dan karuniaNya, laporan tugas akhir yang berjudul \thetitle{} ini dapat dirampungkan dengan baik dan tepat waktu. Penulis juga mengucapkan terima kasih kepada semua pihak yang telah terlibat dan membantu penulis dalam pengerjaan tugas akhir ini, termasuk tapi tidak terbatas padanya diantaranya:

\begin{enumerate}
    \item Bapak Dr. Judhi Santoso, M.Sc. sebagai pembimbing penulis yang telah membimbing dan mengarahkan penulis dalam pengembangan solusi dan penulisan tugas akhir ini.
    \item Bapak Ir. Rila Mandala, M.Eng., Ph.D. sebagai penguji penulis yang telah memberi kritik dan saran untuk tugas akhir ini sehingga dapat menjadi lebih baik lagi.
    \item Ibu Ginar Santika Niwanputri, S.T., M.Sc., Bapak Adi Mulyanto, S.T., M.T., Bapak Nugraha Priya Utama, S.T., M.A., Ph.D., Ibu Latifa Dwiyanti, S.T., M.T., dan seluruh anggota tim tugas akhir lainnya yang telah mengatur dan mengarahkan penulis dan mahasiswa lainnya dalam pengerjaan dan penulisan tugas akhir ini.
    \item Dosen-dosen Teknik Informatika dan dosen Institut Teknologi Bandung secara umum yang telah mendidik dan memberikan ilmu kepada penulis sehingga penulis dapat mengerjakan tugas akhir ini.
    \item Kementerian Pendidikan dan Kebudayaan Indonesia yang telah membiayai penulis dalam menempuh pendidikan tinggi di Teknik Informatika ini.
    \item Keluarga penulis yang selalu mendukung penulis di dalam pengerjaan tugas akhir ini maupun di luarnya.
    \item Teman-teman penulis yang tergabung dalam Tim DAGOZILLA ITB yang telah mengajarkan penulis dalam bidang robotika dan membantu pengerjaan tugas akhir ini baik secara moril maupun materiil.
    \item Teman-teman penulis di dalam atau luar Teknik Informatika yang telah menemani dan mendukung penulis selama pengerjaan tugas akhir ini.
\end{enumerate}

Penulis berharap agar laporan tugas akhir ini dalam segala kekurangannya dapat memberi manfaat untuk seluruh pihak yang membaca maupun terlibat dalam pembuatannya.

\begin{flushright}
    Bandung, \thedate \\[3\baselineskip]
    Penulis
\end{flushright}
